% Einleitung
\section{Einleitung}
VisAQ ist eine Webanwendung, um Luftqualitätsdaten eines heterogenen \gls{Sensor}netzwerks für eine breite Nutzergruppe zu visualisieren.
Ein solches heterogenes Sensornetzwerk kann aus Sensoren unterschiedlicher Hersteller und Qualität bestehen.
So kann es beispielsweise von amtlichen Messstationen bis hin zu einem DIY-Sensor der 50\euro-Klasse alle Sensortypen in einer Datenbank vereinen.
Auch muss nicht jeder Sensor die gleichen Parameter erfassen oder im gleichen Abstand messen.
Dafür wird auf die Daten des Projektes SmartAQNet\footnote{\url{https://www.smartaq.net}} zurückgegriffen.
SmartAQNet sammelt Sensordaten aus dem Großraum Augsburg und bündelt diese in einer Datenbank nach dem Sensorthings-Standard\footnote{\url{https://developers.sensorup.com/docs/}}.
\\
Das Projekt wird im Rahmen des Softwareprojektes PSE am Karlsruher Institut für Technologie durchgeführt.
\\
Auf Basis des Pflichtenhefts
\footnote{\url{https://github.com/ys3d/PSE2020-Luftqualitaetsdaten-Pflichtenheft}}\footnote{\url{https://visaq.de/files/VisAQ_Pflichtenheft.pdf}},
des Entwurf
\footnote{\url{https://github.com/ys3d/PSE2020-Luftqualitaetsdaten-Entwurf}}\footnote{\url{https://visaq.de/files/VisAQ_Entwurf.pdf}}
sowie der Implementierung
\footnote{\url{https://github.com/ys3d/PSE2020-Luftqualitaetsdaten-VisAQ-Backend}}\footnote{\url{https://github.com/ys3d/PSE2020-Luftqualitaetsdaten-VisAQ-Frontend}}\footnote{\url{https://visaq.de/files/VisAQ_Implementierung.pdf}}
VisAQ implementiert.
