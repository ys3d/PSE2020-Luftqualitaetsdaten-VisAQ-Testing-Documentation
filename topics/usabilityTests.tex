\section{Nutzerstudien}
\subsection{Think Aloud}
Im Rahmen der Nutzerstudien fand eine Think Aloud Studie statt. Hierbei wurden sieben Testpersonen unabhängig voneinander Aufgaben gestellt,
welche diese auf der Webseite bewältigen sollten. Dabei sollten die Testpersonen ihre Gedanken und Tätigkeiten laut mit dem Leiter des Testes teilen.
In Anbetracht gezogen wurden hierbei die technischen Kenntnisse der Testpersonen und das benutze Endgerät mit Browser. Im folgenden werden die Ergebnisse zusammengefasst.
Gestellte Aufgaben:

\textbf{Finden Sie einen beliebigen Sensor und finden Sie die aktuellen Luftfeuchtigkeitswerte dieses Sensors heraus.}

Bei dieser Aufgabe haben sich alle Testpersonen gleich verhalten. 
Aufgabe erfüllt durch Klicken auf einen der Sensoren in Augsburg.

\textbf{Finden Sie den Sensor in Gamisch-Partenkirchen und finden Sie heraus welche Werte dieser Sensor misst.}

Bei dieser Aufgabe haben sich alle Testpersonen gleich verhalten. 
Aufgabe erfüllt durch Benutzen der Suchleiste, Klicken auf Suchen und anschließendes Klicken auf den Sensor.

\textbf{Aktivieren Sie die Hilfe und den Experten-Modus.}

Die Hilfe wurde von den Testpersonen nach kurzem Suchen gefunden.
Bei der Aktivierung des Experten-Modus war unklar was dieser macht. Nach einigem Suchen haben die Testpersonen erkannt, dass dieser in der Sensoroverview weitere Daten anzeigt.
Aufgrund dessen wurden an einigen Stellen auf der Webseite weitere Hilfen verteilt, welche Unklarheiten beseitigen sollen. 

\textbf{Finden Sie den berechneten/interpolierten Wert für die Temperatur an einem beliebigen Punkt heraus.}

Unklarheiten durch das Benutzen des Wortes \enquote{Interpoliert}. Testpersonen wussten nicht was gemeint war. Nach einer kurzen Aufklärung war Ihnen bewusst
was gemeint ist. 
Zuerst gesucht in dem \enquote{Weitere Funktionen}-dropdown. Das Interpolationsoverlay wurde dann in Ansichten gefunden.
Aufgrund dessen wurde das interpolierte Overlay umbenannt. 

\subsection{Zusammenfassung des Think Aloud}
Durch die Auswertung der Think Aloud Benutzerstudie wurden Änderungen bezüglich der Benennung von Funktionen durchgeführt. Außerdem wurden mehr Hilfe Popups hinzugefügt damit auch Benutzer mit weniger
technischen Kenntnissen die Seite problemlos benutzen können.

\subsection{Nutzerstudie nach dem System-Usability-Scale-Prinzip}
Die Nutzerstude wurde mit 12 Testpersonen durchgeführt, welche an die System-Usability-Scale Nutzerstudie angelehnt wurde. 
Bei den Teilnehmern der Nutzerstude handelt es sich um Freunde und Bekannte mit unterschiedlichen technischen Kenntnissen. 
Hierbei wurde die allgemeine Benutzbarkeit der Webseite, sowie die Ladezeiten betrachtet und ausgewertet.
Bei beiden Einschätzungen wurde eine Skala von 1 bis 5 verwendet, wobei 1 für \enquote{Stimme überhaupt nicht zu} und 5 für \enquote{Stimme voll zu} steht.

Im folgenden werden die Ergebnisse vorgestellt.
\subsubsection{Allgemeine Benutzung der Seite}
\begin{figure}[H]
    \centering
    \includegraphics[width=0.7\textwidth]{media/survey/regularUsage.png}
\end{figure}
Die weniger positiven Bewertungen stammen vermutlich davon, dass weniger Interesse bezüglich Luftqualitätsdaten besteht.

\begin{figure}[H]
    \centering
    \includegraphics[width=0.7\textwidth]{media/survey/complexity.png}
\end{figure}

\noChanges

\begin{figure}[H]
    \centering
    \includegraphics[width=0.7\textwidth]{media/survey/easyUsage.png}
\end{figure}

\noChanges

\begin{figure}[H]
    \centering
    \includegraphics[width=0.7\textwidth]{media/survey/technicalSupport.png}
\end{figure}

\noChanges

\begin{figure}[H]
    \centering
    \includegraphics[width=0.7\textwidth]{media/survey/integrity.png}
\end{figure}

\noChanges

\begin{figure}[H]
    \centering
    \includegraphics[width=0.7\textwidth]{media/survey/inconsistent.png}
\end{figure}

\noChanges

\begin{figure}[H]
    \centering
    \includegraphics[width=0.7\textwidth]{media/survey/fastLearning.png}
\end{figure}

\noChanges
\begin{figure}[H]
    \centering
    \includegraphics[width=0.7\textwidth]{media/survey/overelaboratedInterface.png}
\end{figure}

\noChanges
\begin{figure}[H]
    \centering
    \includegraphics[width=0.7\textwidth]{media/survey/safeUsage.png}
\end{figure}

\noChanges
\begin{figure}[H]
    \centering
    \includegraphics[width=0.7\textwidth]{media/survey/needToLearnUsage.png}
\end{figure}

\noChanges

Zusammengefasst ergibt sich ein gutes Ergebnis bezüglich der allgemeinen Benutzung der Seite und keine ausschlaggebenden Änderungen.
\subsubsection{Allgemeine Laufzeit der Seite}
Hierbei wurde nach der allgemeinen Warte- und Laufzeit gefragt. Diese wurde von den Teilnehmern als nicht störrend oder unangenehm lang wahrgenommen.
Zusammengefasst ergab sich für die drei Fragen hierbei ein sehr gutes Bild. 

\subsubsection{Freie Antwortmöglichkeiten}
Die freien Antwortmöglichkeiten ergaben, dass die Nutzer zufrieden mit der Webseite sind. 
Eine Antwort ergab, dass es eine Barriere bezüglich Farbenblindheit gab.
Dies führte dazu, dass der in den Wunschkriterien erwähnte und später verworfene Farbenblindenmodus doch einführt wurde. 

\subsection{Zusammenfassung der Nutzerstudie nach System-Usability-Scale}

Die Nutzerstudie ergab das wenig Änderungen an der Webseite durchgeführt werden mussten.
