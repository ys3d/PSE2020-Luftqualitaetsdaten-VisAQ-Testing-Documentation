\subsection{Frontend}
Bei den Komponententests des Frontends wurde der Fokus auf den Bereich der Controller und des Models gelegt.
Zudem wurden auch einige Komponenten der View getestet.

Der einzige Controller im Frontend ist der Request-Controller.
Bereits in der Implementierungsphase wurde ein grundlegender Test für den Request-Controller entwickelt.
Dieser Test wurde so erweitert, dass er nun beinahe alle möglichen Fälle des Request-Controllers umfasst.
Der Einzige nicht getestete Fall ist die Wahl des Debugging-Modus.
Im Debugging-Modus wird durch den Request-Controller auf ein Backend mit der Adresse http://localhost:8080 zugegriffen, dies wird nicht getestet, da nicht immer gewährleistet werden kann, dass auf dem Testsystem gleichzeitig ein entsprechendes Backend aktiv ist.

Die Komponententests im Model wurden erst in der Testphase ergänzt.
Die Tests decken 100\% des Programmcodes ab.
Alle Modelle des Frontends wurden nach dem gleichen Schema getestet.
Beispielhaft sieht ein Test folgendermassen aus:
\begin{lstlisting}[style=customjs]
test("HistoricalLocation Parse", () => {
    const hl = new HistoricalLocation([...]);

    expect(hl.time).toBe('2019-07-02T15:23:53.749Z');
    expect(hl.thingLink.url).toBe("https://api.smartaq.net/v1.0/HistoricalLocations('66c68826-9cdd-11e9-a024-ffbd2647ca86')/Thing");
    expect(hl.locationsLink.url).toBe("https://api.smartaq.net/v1.0/HistoricalLocations('66c68826-9cdd-11e9-a024-ffbd2647ca86')/Locations");

    const hl2 = new HistoricalLocation(hl.toJSON());
    expect(hl2).not.toBe(undefined);
    expect(hl2.time).toBe(hl.time);
    expect(hl2.thingLink.url).toBe(hl.thingLink.url);
    expect(hl2.locationsLink.url).toBe(hl.locationsLink.url);
});
\end{lstlisting}
Bei [...] steht ein HistoricalLocation-Objekt im \gls{JSON}-Format wie es vom Backend geliefert wird.
\\
Der Test prüft zuerst, ob eine HistoricalLocation-Instanz korrekt aus dem \gls{JSON}-Format gelesen wird.
Anschließend wird die \textit{toJSON()}-Funktion aufgerufen und aus dem entstandenen JSON-Objekt wieder eine HistoricalLocation-Instanz erstellt.
Diese entstandene Instanz wird dann auf die Übereinstimmung mit der ersten Instanz überprüft.

Im View-Teil des Frontends wurden nicht alle Komponenten mit automatisierten Tests erfasst.
Die tatsächlich anzeigenden Komponenten wurden durch die Nutzertests und die Benutzung durch das Entwicklerteam getestet.
Im View-Teil befinden sich zusätzlich einige Singletons, die zur Zustandsverwaltung der Anzeige dienen.
Diese und ihre Interaktion mit den Cookies wurde mit einer Überdeckung von 100\% getestet.
\\
Die rein anzeigenden Komponenten der View wurden nicht mit Komponententests überdeckt. 
