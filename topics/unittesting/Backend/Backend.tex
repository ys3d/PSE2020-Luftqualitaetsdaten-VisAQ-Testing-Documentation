\subsection{Backend}
Im Backend werden JUnit\footnote{\url{https://junit.org/}} Tests verwendet. Es werden hierbei Tests für die JUnit Version 5 geschrieben.
Um die Qualität und die Code-Abdeckung der Testfälle zu beurteilen, wird das System Pitest\footnote{\url{https://pitest.org/}} verwendet.
Pitest führt Mutation Testing\footnote{\url{https://en.wikipedia.org/wiki/Mutation_testing}} durch.
Bei dieser Methode werden Testfälle generiert, indem der Programmcode an je einer Stelle pro Durchlauf verändert wird.
Nun sollte mindestens einer der Testfälle fehlschlagen.
Ist dies nicht der Fall, so wurde die veränderte Codezeile nicht hinreichend genau getestet.
Allerdings können im Allgemeinen nicht alle Mutanten abgefangen werden.
Zusätzlich generiert Pitest einen umfangreichen Testbericht und erfasst auch die reine Zeilenüberdeckung der Tests.

Bereits in der Implementierungsphase wurde ein großer Teil der Testfälle erstellt.
Verbesserungen im Programmcode und teilweise Ungenauigkeiten in den Tests haben zu den Erweiterungen in den Testfällen geführt.

Die Überdeckungsergebnisse des finalen Teststandes können unter \url{backendtest2.visaq.de} abgerufen werden.
Im Folgenden sollen die Ergbnisse und Methoden kurz erläutert werden

\subsubsection{Nicht getestete Code-Teile}
Im Testbericht des Backends werden 4 Zeilen Code und 16 Mutanten als nicht getestet gekennzeichnet.
Hierzu sollen nun kurz die Gründe für die fehlende Überdeckung geschildert werden.

\paragraph{DatastreamController}
\begin{lstlisting}[style=customjava]
@Override
public ArrayList<Datastream> getAll() {
    return new MultiOnlineLink<Datastream>("/Datastreams", true).get(this);
}
\end{lstlisting}
\textbf{Ungetestet: 1 Zeile 1 Mutant}
\\
Die Funktion getAll für Datastreams soll alle Datastreams aus der Datenbank zu laden.
Aufgrund der Menge der in der Datenbank vorhandenen Datastream-Objekte wäre ein Test der Funktionalität hierbei nicht effizient möglich.
Daher wurde hier auf einen Test verzichtet.

\paragraph{ObservationController}
\begin{lstlisting}[style=customjava]
@Override
public ArrayList<Observation> getAll() {
    return new MultiOnlineLink<Observation>("/Observations", true).get(this);
}
\end{lstlisting}
\textbf{Ungetestet: 1 Zeile 1 Mutant}
\\
Beim ObservationController tritt ebenfalls das Problem der getAll-Funktion auf.
Auch hier wurde aufgrund der Menge an Observations kein Test geschrieben.

\begin{lstlisting}[style=customjava]
try {
    es.awaitTermination(30, TimeUnit.SECONDS);
} catch (InterruptedException e) {
    throw new ResponseStatusException(HttpStatus.INTERNAL_SERVER_ERROR);
}
\end{lstlisting}
\textbf{Ungetestet: 1 Zeile}
\\
Eine Version der getAll-Funktion mit Parametern im ObservationController verwendet Multithreading, um die Laufzeit der Operation zu verbessern, da diese oft aufgerufen wird und verhältnismäßig zeitintensiv ist.
Für den Fall das ein Thread erheblich gestört wird, wird hier ein Timeout von 30 Sekunden gesetzt. Im Falle eines Timeouts wird ein serverinterner Fehler von Spring ausgelöst.
Die Fehlerauslösung wird nicht von den Testfällen abgedeckt.

\paragraph{DefaultInterpolation}
\begin{lstlisting}[style=customjava]
bsi.setLengthScale(0.02);
bsi.setMinObservationCount(0);
double invSqrt2 = 1 / Math.sqrt(2);
double maxObservationDistance = invSqrt2 * Math.max(Math.abs(trans.transformX(1) - trans.transformX(0)), Math.abs(trans.transformY(1) - trans.transformY(0)));
bsi.setMaxObservationDistance(maxObservationDistance);
bsi.setNoData(-99999);
bsi.setPassCount(1);
\end{lstlisting}
\textbf{Ungetestet: 4 Mutant}
\\
Pitest prüft beim Aufrufen von Funktionen ohne Rückgabewert, ob ein Entfernen der Funktionsaufrufe zu Fehlern führt.
Bei 4 der Konfigurationsaufrufe an die Interpolations-Bibliothek wird dieser Mutant nicht abgedeckt.
Um die Aufrufe zu testen benötigt man Powermock\footnote{Siehe \url{https://de.wikipedia.org/wiki/PowerMock}}, um den Konstruktor der Interpolationsklasse zu mocken, jedoch ist Powermock noch nicht mit Junit 5 kompatibel.

\paragraph{GridTransform}
\begin{lstlisting}[style=customjava]
public double transformX(int i) {
    if (i >= xcoord - 1) {
        return env.getMaxX();
    }
    return env.getMinX() + i * dx;
}
\end{lstlisting}
\textbf{Ungetestet: 2 Mutant}
\\
In der Klasse GridTransform sind 2 Mutanten ungetestet.
Hierbei handelt es sich um den Mutanten, der \textit{if (i >= xcoord - 1)} zu \textit{if (i > xcoord - 1)} verändert.
Dieser Mutant kann nicht getestet werden, da der Effekt durch das zweite Return-Statement aufgehoben wird.
Der selbe Mutant kommt ebenfalls in der Funktion \textit{public double transformY(int j)} vor.

\paragraph{NearestNeighborInterpolation}
\mbox{}\\
\textbf{Ungetestet: 1 Zeile 6 Mutant}
\\
Zum Teil können diese nicht getestet werden, zum Teil wurden aber auch keine weiteren Tests mehr geschrieben, da die NearestNeighborInterpolation nichtmehr aktiv verwendet wird.

